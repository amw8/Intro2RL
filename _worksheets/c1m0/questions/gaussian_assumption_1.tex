In real world problems, we rarely know the distribution of a variable. Instead, we often make assumptions on its distribution. One commonly assumed distribution for continuous random variables is the Gaussian (or Normal) distribution. List at least two reasons that justifies using the Gaussian distribution in practice.

\smallspace

%% \textbf{Answer:}
%% \begin{enumerate}
%% \item Because of Central Limit Theorem (distribution of sample mean, i.e. mean of random variables drawn from arbitrary distributions, approaches a Normal distribution as the number of samples increases). If we assume that a real world process has many underlying contributing factors (coming from arbitrary distributions), then a Normal distribution is likely to model it well.
%% \item Because of simplicity of the Normal distribution: it just has two parameters $\mu$ and $\sigma$, and has nice properties, for example it is a conjugate distribution.
%% \end{enumerate}
