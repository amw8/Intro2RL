Consider the following game: you roll two (fair) 6-sided dice and win $\$1$ if the sum of the dice roll is 2, 5, 7, 8 or 11. Otherwise, you lose $\$1$.

\begin{enumerate}[label=(\alph*)]
    \item What is the expected value of the sum of the two dice?
    \item What is the variance of the sum of the two dice?
    \item What is the expected value of the winnings for playing this game. In other words, how much money are you expected to gain (or lose).
    \item What is the variance of the winnings for this game?
    \item Would you play this game as stated above? How about if the amount won or lost was $\$100$? How about $\$1000$?
\end{enumerate}

\smallspace

%% \textbf{Answer:}
%% \begin{enumerate}
%% \item  Let $X_1$ and $X_2$ denote the random variables representing the value on the first and second roll of the dice respectively.
%%   \begin{equation*}
%%     \E[X_1 + X_2] = \E[X_1] + \E[X_2] = 2 \E[X_1] = 2 \times \frac{1}{6} \times (1 + 2 + 3 + 4 + 5 + 6) = 2 \times 3.5 = 7.
%%   \end{equation*}

%% \item For instructive purpose let us analyze the variance of the sum of the two dice first:
%%   \begin{IEEEeqnarray*}{lCl}
%%     \text{Var}[X_1 + X_2] &:=& \E\left[\Big(X_1+X_2 - \E[X_1+X_2]\Big)^2\right] \\
%%     &=& \E[(X_1+X_2)^2] - \E[X_1+X_2]^2 \\
%%     &=& \E[X_1^2 + X_2^2 + 2X_1X_2] - \E[X_1]^2 - \E[X_2]^2 - 2\E[X_1]\E[X_2] \\
%%     &=& \Big(\E[X_1^2] - \E[X_1]^2\Big) + \Big(\E[X_2^2] - \E[X_2]^2\Big) + 2 \Big(\E[X_1X_2] - \E[X_1]\E[X_2]\Big) \\
%%     &=& \text{Var}[X_1] + \text{Var}[X_2] + 2 \text{Cov}(X_1, X_2) \\
%%     &=& \text{Var}[X_1] + \text{Var}[X_2]. \hfill \text{(since $X_1$ and $X_2$ are independent)} 
%%   \end{IEEEeqnarray*}

%%   Now continuing a similar analysis we did before, we get
%%   \begin{IEEEeqnarray*}{lCl}
%%     \text{Var}[X_1 + X_2] &=& \text{Var}[X_1] + \text{Var}[X_2] =2 \text{Var}[X_1] \\
%%     &=& 2  \Big(\E[X_1^2] - \E[X_1]^2\Big)\\
%%     &=& 2 \times \left(\frac{1}{6} \times (1^2 + 2^2 + 3^2 + 4^2 + 5^2 + 6^2) - 3.5^2\right) \\
%%     &=& 5.83.    
%%   \end{IEEEeqnarray*}

%% \item We have the following combinations for the digits on the dice for which we gain $\$1$:
%%   \begin{itemize}
%%   \item $X_1 + X_2 = 2: \quad \{(1, 1)\}$,
%%   \item $X_1 + X_2 = 5: \quad\{(1, 4), (2, 3), (3, 2), (4, 1)\}$,
%%   \item $X_1 + X_2 = 7: \quad\{(1, 6), (2, 5), (3, 4), (4, 3), (5, 2), (6, 1)\}$,
%%   \item $X_1 + X_2 = 8: \quad\{(2, 6), (3, 5), (4, 4), (5, 3), (6, 2)\}$, and
%%   \item $X_1 + X_2 = 11: \;\; \{(5, 6), (6, 5)\}$.
%%   \end{itemize}
%%   With this information, the expectation is straightforward to calculate. If we let $W$ represent the random variable denoting the amount of winnings:
%%   \begin{equation*}
%%     \E[W] = \frac{18}{36} \times (+1) + \frac{18}{36} \times (-1) = 0.
%%   \end{equation*}

%% \item Using the above information we get
%%   \begin{equation*}
%%     \text{Var}[W] = \E[W^2] - \E[W]^2 = \left(\frac{18}{36} \times (+1)^2 + \frac{18}{36} \times (-1)^2\right) - 0^2 = 1.
%%   \end{equation*}

%% \item It is equally likely for us to win or lose the money in this game -- so it is purely dependent on luck. It would be more wise to play this game for a smaller amount if we want to minimize the risk of losing money. On the flip side, playing with a smaller amount would also decrease the potential gains.
  
%% \end{enumerate}
