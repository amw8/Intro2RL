(\textit{Exercise 9.2 S\&B})
Recall: each state $s$ corresponds to $k$ numbers, $s_{1}, s_{2}, \dotso , s_{k}$ with each $s_{i} \in \mathbb{R}$. For this $k-$dimensional state space, each order-$n$ polynomial-basis feature $x_{i}$ can be written as
$$x_{i}(s) = \prod_{j=1}^{k} s^{c_{i,j}}_{j}$$
where each $c_{i,j}$ is an integer in the set $\{0,1,\dotso, n\}$ for an integer $n \geq 0$. These features make up the order-$n$ polynomial basis for dimension k, which contains $(n+1)^{k}$ different features


Why does the above equation have $(n + 1)^{k}$ distinct features for dimension $k$?